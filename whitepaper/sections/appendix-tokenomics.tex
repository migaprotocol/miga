% =============================================================================
\appendix
\section{Appendix A: Complete Token Economics}
% =============================================================================

\subsection{Token Hierarchy Overview}

The MIGA protocol implements a five-token system designed to balance capital, participation, commitment, reputation, and governance power. Each token serves a distinct purpose in the economic and governance mechanism.

\begin{figure}[h]
\centering
\begin{tcolorbox}[colback=white,colframe=migablue,width=0.95\textwidth,title={\textbf{Token Hierarchy Diagram}}]
\begin{center}
\texttt{
\begin{tabular}{c}
\\[0.5em]
\textbf{ASHA} (Governance Power) \\
$\Downarrow$ \\
Combines all factors below \\[1em]
\hline
\\[0.5em]
\textbf{CYRUS} (Foundation Collateral) \\
Store of value, long-term commitment \\[0.5em]
+ \\[0.5em]
\textbf{MIGA} (Community Token) \\
Participation, community alignment \\[0.5em]
+ \\[0.5em]
\textbf{vePARS} (Vote-Escrowed PARS) \\
Time-locked commitment \\[0.5em]
$\times$ \\[0.5em]
\textbf{KARMA} (Reputation Score) \\
Soulbound, non-transferable \\[1em]
\hline
\\[0.5em]
$\Downarrow$ \\
Final governance power for proposals
\end{tabular}
}
\end{center}
\end{tcolorbox}
\caption{Token hierarchy showing how different factors combine to create governance power}
\end{figure}

% =============================================================================
\subsection{Token Definitions and Purposes}
% =============================================================================

\begin{longtable}{p{2.5cm}p{2.5cm}p{7.5cm}}
\toprule
\textbf{Token} & \textbf{Type} & \textbf{Purpose \& Properties} \\
\midrule
\endhead

\textbf{CYRUS} & Foundation Collateral &
\begin{itemize}[leftmargin=*,nosep]
\item Store of value representing long-term commitment
\item ERC-20 token on Ethereum mainnet
\item Foundation treasury backing
\item Weighted 30\% in governance formula
\item Prevents purely speculative governance
\end{itemize} \\[1em]

\textbf{MIGA} & Community Token &
\begin{itemize}[leftmargin=*,nosep]
\item Primary participation and community token
\item SPL token on Solana, bridged to 6 chains
\item 1B total supply, fair launch via bonding curve
\item Required for vePARS creation (paired lock)
\item Weighted 50\% in governance formula (as $\sqrt{M \cdot V}$)
\end{itemize} \\[1em]

\textbf{PARS} & Emission Token &
\begin{itemize}[leftmargin=*,nosep]
\item Rewards for builders and contributors
\item Emitted continuously, decays when idle
\item Must be locked to create vePARS
\item Rebasing mechanism (sPARS) for continuous participation
\item Foundation for vote-escrowed governance power
\end{itemize} \\[1em]

\textbf{vePARS} & Vote-Escrow &
\begin{itemize}[leftmargin=*,nosep]
\item Non-transferable governance power
\item Created by locking PARS + MIGA together
\item Time-weighted: longer locks = more power
\item Diminishing returns via square root scaling
\item Weighted 20\% in final ASHA formula
\end{itemize} \\[1em]

\textbf{KARMA} & Reputation Score &
\begin{itemize}[leftmargin=*,nosep]
\item Soulbound reputation (0-1 scale)
\item Non-transferable, tied to identity
\item Earned through verified contributions
\item Multiplicative factor in governance
\item Prevents Sybil attacks and pure capital dominance
\end{itemize} \\[1em]

\textbf{ASHA} & Governance Power &
\begin{itemize}[leftmargin=*,nosep]
\item Computed value, not a tradeable token
\item Combines all factors above via formula
\item Used for proposal voting and governance
\item Updated dynamically based on holdings and reputation
\item Name means "truth/righteousness" in Persian
\end{itemize} \\

\bottomrule
\end{longtable}

% =============================================================================
\subsection{The ASHA Governance Power Formula}
% =============================================================================

\subsubsection{Complete Formula}

The ASHA governance power for user $u$ is computed as:

\begin{equation}
\boxed{
\text{ASHA}(u) = \left( 0.3 \cdot C_u + 0.5 \cdot \sqrt{M_u \cdot V_u} + 0.2 \cdot V_u \right) \times \left(1 + \ln(1 + K_u \cdot e)\right)
}
\end{equation}

\textbf{Where:}

\begin{itemize}
\item $C_u$ = CYRUS holdings (normalized to [0, 1] by max holdings)
\item $M_u$ = MIGA holdings (normalized to [0, 1] by max holdings)
\item $V_u$ = vePARS holdings (normalized to [0, 1] by max vePARS)
\item $K_u$ = KARMA score $\in [0, 1]$ (soulbound reputation)
\item $e \approx 2.718$ = Euler's constant
\end{itemize}

\subsubsection{Component Analysis}

\textbf{1. Base Power (Linear Combination):}

\begin{equation}
\text{Base}(u) = 0.3 \cdot C_u + 0.5 \cdot \sqrt{M_u \cdot V_u} + 0.2 \cdot V_u
\end{equation}

The base power combines three factors:

\begin{itemize}
\item \textbf{CYRUS term (30\%):} Linear contribution from foundation collateral
  \begin{itemize}
  \item Rewards long-term commitment to foundation stability
  \item Largest single-token weight, but not majority
  \item Prevents pure MIGA whales from dominating
  \end{itemize}

\item \textbf{Geometric mean term (50\%):} $\sqrt{M_u \cdot V_u}$ enforces balance
  \begin{itemize}
  \item Requires \textit{both} MIGA and vePARS for maximum power
  \item Diminishing returns: doubling one token gives only $\sqrt{2} \approx 1.41\times$ boost
  \item Prevents single-axis optimization
  \item Largest weight in formula, emphasizing community + commitment
  \end{itemize}

\item \textbf{vePARS term (20\%):} Additional reward for time-locking
  \begin{itemize}
  \item Linear bonus for long-term commitment
  \item Complements geometric mean term
  \item Rewards those who lock liquidity
  \end{itemize}
\end{itemize}

\textbf{2. KARMA Multiplier (Logarithmic):}

\begin{equation}
\text{Multiplier}(K_u) = 1 + \ln(1 + K_u \cdot e)
\end{equation}

Properties of the KARMA multiplier:

\begin{itemize}
\item \textbf{Range:} $[1, 2]$ for $K_u \in [0, 1]$
  \begin{itemize}
  \item Minimum: $K_u = 0 \Rightarrow$ multiplier $= 1$ (no boost)
  \item Maximum: $K_u = 1 \Rightarrow$ multiplier $= 1 + \ln(1 + e) \approx 2.313$
  \end{itemize}

\item \textbf{Logarithmic scaling:} Diminishing returns on reputation
  \begin{itemize}
  \item Prevents reputation from dominating capital
  \item Going from 0 to 0.5 KARMA adds more relative power than 0.5 to 1.0
  \item Encourages broad participation over perfect scores
  \end{itemize}

\item \textbf{Multiplicative, not additive:} Amplifies existing power
  \begin{itemize}
  \item Cannot gain governance power from reputation alone
  \item Must have capital (CYRUS, MIGA) or commitment (vePARS)
  \item Prevents Sybil attacks while rewarding contribution
  \end{itemize}
\end{itemize}

% =============================================================================
\subsection{vePARS: Vote-Escrowed PARS Formula}
% =============================================================================

\subsubsection{Core Formula}

Vote-escrowed PARS (vePARS) is created by locking PARS + MIGA together:

\begin{equation}
\boxed{
\text{vePARS}(u) = \min(\text{PARS}_u, \text{MIGA}_u) \times \sqrt{\frac{t_{\text{lock}}}{t_{\text{max}}}}
}
\end{equation}

\textbf{Where:}

\begin{itemize}
\item $\text{PARS}_u$ = Amount of PARS tokens locked
\item $\text{MIGA}_u$ = Amount of MIGA tokens locked
\item $t_{\text{lock}}$ = Lock duration chosen by user
\item $t_{\text{max}}$ = Maximum lock duration (e.g., 4 years = 1461 days)
\end{itemize}

\subsubsection{Design Rationale}

\textbf{1. Paired Lock Requirement ($\min$ function):}

\begin{itemize}
\item Must lock \textit{equal amounts} of PARS and MIGA
\item Prevents pure PARS farmers from gaining governance power
\item Ensures community alignment (MIGA) + builder rewards (PARS)
\item Example: Locking 100 PARS + 50 MIGA creates vePARS based on 50 tokens
\end{itemize}

\textbf{2. Time-Weighted Power ($\sqrt{\frac{t}{t_{\text{max}}}}$):}

Square root scaling provides:

\begin{itemize}
\item \textbf{Increasing returns:} Longer locks earn more power
\item \textbf{Diminishing returns:} Not linear, prevents lock-time dominance
\item \textbf{Fairness:} 4-year lock gets $2\times$ power of 1-year, not $4\times$
\end{itemize}

\textbf{Power scaling examples:}

\begin{center}
\begin{tabular}{ccc}
\toprule
\textbf{Lock Duration} & \textbf{Time Fraction} & \textbf{Power Multiplier} \\
\midrule
3 months & $\frac{91}{1461} \approx 0.062$ & $\sqrt{0.062} \approx 0.25\times$ \\
6 months & $\frac{182}{1461} \approx 0.125$ & $\sqrt{0.125} \approx 0.35\times$ \\
1 year & $\frac{365}{1461} \approx 0.25$ & $\sqrt{0.25} = 0.5\times$ \\
2 years & $\frac{730}{1461} \approx 0.50$ & $\sqrt{0.50} \approx 0.71\times$ \\
4 years & $\frac{1461}{1461} = 1.0$ & $\sqrt{1.0} = 1.0\times$ \\
\bottomrule
\end{tabular}
\end{center}

\textbf{Key insight:} Doubling lock time from 1 year to 2 years increases power by only $41\%$, not $100\%$.

\subsubsection{Mathematical Derivation}

\textbf{Why square root instead of linear?}

Consider two lock strategies:

\begin{itemize}
\item \textbf{User A:} Locks 1000 tokens for 4 years
\item \textbf{User B:} Locks 1000 tokens for 1 year, four times consecutively
\end{itemize}

\textbf{Linear time weighting} ($t / t_{\text{max}}$):
\begin{align}
\text{Power}_A &= 1000 \times \frac{4}{4} = 1000 \\
\text{Power}_B &= 4 \times \left( 1000 \times \frac{1}{4} \right) = 1000
\end{align}

Same power! No incentive for long-term commitment.

\textbf{Square root time weighting} ($\sqrt{t / t_{\text{max}}}$):
\begin{align}
\text{Power}_A &= 1000 \times \sqrt{\frac{4}{4}} = 1000 \\
\text{Power}_B &= 4 \times \left( 1000 \times \sqrt{\frac{1}{4}} \right) = 4 \times 500 = 2000
\end{align}

User B gets \textit{more} power by staying flexible? No! We need the square root to be applied to the \textit{time fraction}, not the amount. Recalculate:

\begin{align}
\text{Power}_B \text{ (average per year)} &= 1000 \times \sqrt{\frac{1}{4}} = 500 \text{ per lock} \\
\text{Total over 4 years} &\neq 2000 \text{ (cannot stack simultaneously)}
\end{align}

The key is that vePARS is \textit{non-transferable} and \textit{locked}. User B cannot have four simultaneous 1-year locks. User A gets the full benefit of long-term commitment.

% =============================================================================
\subsection{KARMA Reputation System}
% =============================================================================

\subsubsection{KARMA Score Definition}

KARMA is a normalized reputation score $K \in [0, 1]$ that measures verified contributions to the MIGA ecosystem.

\textbf{Properties:}

\begin{itemize}
\item \textbf{Soulbound:} Tied to identity, non-transferable
\item \textbf{Non-tradeable:} Cannot be bought or sold
\item \textbf{Verified:} Earned through attestations and proofs
\item \textbf{Continuous:} Updated dynamically as contributions are verified
\item \textbf{Privacy-preserving:} ZK proofs allow proving score without revealing identity
\end{itemize}

\subsubsection{KARMA Calculation}

Raw KARMA score is computed from contribution metrics:

\begin{equation}
K_{\text{raw}}(u) = w_1 \cdot \text{Proposals}_u + w_2 \cdot \text{Votes}_u + w_3 \cdot \text{Grants}_u + w_4 \cdot \text{Service}_u
\end{equation}

\textbf{Where:}

\begin{itemize}
\item $\text{Proposals}_u$ = Approved proposals submitted (0-100)
\item $\text{Votes}_u$ = Governance participation rate (0-100)
\item $\text{Grants}_u$ = Grants delivered successfully (0-100)
\item $\text{Service}_u$ = Community service contributions (0-100)
\item $w_i$ = Weights summing to 1 (e.g., [0.3, 0.2, 0.3, 0.2])
\end{itemize}

\textbf{Normalization:} Scores are normalized to [0, 1]:

\begin{equation}
K(u) = \frac{K_{\text{raw}}(u)}{\max_v K_{\text{raw}}(v)}
\end{equation}

\subsubsection{KARMA Multiplier Function}

In the ASHA formula, KARMA acts as a \textit{multiplicative boost}:

\begin{equation}
M(K) = 1 + \ln(1 + K \cdot e)
\end{equation}

\textbf{Analysis of multiplier properties:}

\begin{center}
\begin{tabular}{cccc}
\toprule
\textbf{KARMA} & \textbf{$K \cdot e$} & \textbf{$\ln(1 + K \cdot e)$} & \textbf{Multiplier} \\
\midrule
0.0 & 0 & 0 & 1.00 \\
0.2 & 0.544 & 0.434 & 1.43 \\
0.4 & 1.087 & 0.736 & 1.74 \\
0.6 & 1.631 & 0.966 & 1.97 \\
0.8 & 2.175 & 1.157 & 2.16 \\
1.0 & 2.718 & 1.313 & 2.31 \\
\bottomrule
\end{tabular}
\end{center}

\textbf{Key properties:}

\begin{itemize}
\item \textbf{Minimum boost:} Even $K = 0$ gives multiplier $= 1$ (no penalty)
\item \textbf{Maximum boost:} $K = 1$ gives multiplier $\approx 2.31$ (131\% increase)
\item \textbf{Diminishing returns:} Going from 0.0 to 0.5 adds 0.74x, but 0.5 to 1.0 adds only 0.57x
\item \textbf{Logarithmic:} Prevents reputation from dominating capital factors
\end{itemize}

% =============================================================================
\subsection{Governance Tiers}
% =============================================================================

The ASHA score determines governance capabilities:

\begin{longtable}{p{2cm}p{2cm}p{3cm}p{5.5cm}}
\toprule
\textbf{Tier} & \textbf{ASHA Range} & \textbf{Title} & \textbf{Capabilities} \\
\midrule
\endhead

0 & 0 - 10 & Observer &
\begin{itemize}[leftmargin=*,nosep]
\item View proposals
\item Read governance forums
\item Participate in community discussions
\end{itemize} \\[1em]

1 & 10 - 50 & Participant &
\begin{itemize}[leftmargin=*,nosep]
\item Vote on proposals
\item Comment on governance forums
\item Signal preferences (non-binding)
\end{itemize} \\[1em]

2 & 50 - 100 & Contributor &
\begin{itemize}[leftmargin=*,nosep]
\item Submit improvement proposals (PIPs)
\item Propose grant allocations
\item Vote with full weight
\end{itemize} \\[1em]

3 & 100 - 250 & Guardian &
\begin{itemize}[leftmargin=*,nosep]
\item Submit constitutional amendments
\item Propose new DAOs
\item Vote on Supreme Council decisions
\end{itemize} \\[1em]

4 & 250+ & Elder &
\begin{itemize}[leftmargin=*,nosep]
\item Emergency pause authority (requires 5 Elders)
\item Protocol parameter changes
\item Supreme Council membership eligibility
\end{itemize} \\

\bottomrule
\end{longtable}

\textbf{Note:} ASHA scores are not static tiers but continuous values. Tiers provide intuitive ranges for different governance rights.

% =============================================================================
\subsection{Economic Flywheel}
% =============================================================================

The MIGA token economics create a self-reinforcing economic flywheel:

\begin{figure}[h]
\centering
\begin{tcolorbox}[colback=white,colframe=migadark,width=0.95\textwidth,title={\textbf{Economic Flywheel Mechanism}}]
\begin{center}
\begin{tabular}{c}
\\[0.5em]
\textbf{1. Community Participation} \\
Users buy MIGA on bonding curve \\
$\Downarrow$ \\[0.5em]
\textbf{2. Treasury Growth} \\
50\% of funds go to DAO treasury \\
$\Downarrow$ \\[0.5em]
\textbf{3. Grant Funding} \\
Treasury funds grants and programs \\
PARS emissions reward builders \\
$\Downarrow$ \\[0.5em]
\textbf{4. Lock for Governance} \\
Builders lock PARS + MIGA for vePARS \\
Long-term holders lock for governance power \\
$\Downarrow$ \\[0.5em]
\textbf{5. Liquidity Lock} \\
Locked tokens reduce circulating supply \\
Price appreciation from reduced supply \\
$\Downarrow$ \\[0.5em]
\textbf{6. Reputation Growth} \\
Contributors earn KARMA through delivery \\
KARMA amplifies governance power \\
$\Downarrow$ \\[0.5em]
\textbf{7. Better Governance} \\
High-ASHA participants make better decisions \\
Protocol value increases \\
$\Downarrow$ \\[0.5em]
\textbf{8. Attracts More Participants} \\
Success attracts new community members \\
Cycle repeats \\[0.5em]
\end{tabular}
\end{center}
\end{tcolorbox}
\caption{Self-reinforcing economic flywheel driven by participation, contribution, and governance}
\end{figure}

\subsubsection{Key Feedback Loops}

\textbf{1. Supply Contraction Loop:}

\begin{equation}
\text{vePARS locks} \rightarrow \downarrow \text{Supply} \rightarrow \uparrow \text{Price} \rightarrow \uparrow \text{Treasury Value} \rightarrow \uparrow \text{Grant Funding}
\end{equation}

\textbf{2. Reputation Loop:}

\begin{equation}
\text{Grants} \rightarrow \text{Delivery} \rightarrow \uparrow \text{KARMA} \rightarrow \uparrow \text{Governance Power} \rightarrow \text{Better Proposals}
\end{equation}

\textbf{3. Commitment Loop:}

\begin{equation}
\text{Long locks} \rightarrow \uparrow \text{vePARS} \rightarrow \uparrow \text{ASHA} \rightarrow \text{Governance Influence} \rightarrow \text{Protocol Success}
\end{equation}

\subsubsection{Anti-Capture Mechanisms}

The token design includes multiple mechanisms to prevent governance capture:

\begin{enumerate}
\item \textbf{Paired lock requirement:} Cannot dominate with PARS alone
\item \textbf{Geometric mean term:} Diminishing returns on single-token accumulation
\item \textbf{KARMA multiplier:} Reputation required for maximum power
\item \textbf{Square root time scaling:} Long locks rewarded but not linearly
\item \textbf{Logarithmic KARMA:} Reputation cannot dominate capital
\item \textbf{Multi-token requirement:} Need CYRUS, MIGA, PARS, and KARMA
\end{enumerate}

% =============================================================================
\subsection{Numerical Examples}
% =============================================================================

\subsubsection{Example 1: Pure Capital (Whale)}

\textbf{User profile:}
\begin{itemize}
\item CYRUS: 1,000,000 (normalized to $C = 0.8$)
\item MIGA: 10,000,000 (normalized to $M = 0.9$)
\item vePARS: 0 (no lock, $V = 0$)
\item KARMA: 0 (no contributions, $K = 0$)
\end{itemize}

\textbf{ASHA calculation:}
\begin{align}
\text{Base} &= 0.3 \cdot 0.8 + 0.5 \cdot \sqrt{0.9 \cdot 0} + 0.2 \cdot 0 \\
&= 0.24 + 0 + 0 = 0.24 \\
\text{Multiplier} &= 1 + \ln(1 + 0 \cdot e) = 1.0 \\
\text{ASHA} &= 0.24 \times 1.0 = \boxed{0.24}
\end{align}

\textbf{Result:} Despite massive capital, whale gets only 24\% power due to no lock and no reputation.

\subsubsection{Example 2: Balanced Contributor}

\textbf{User profile:}
\begin{itemize}
\item CYRUS: 10,000 (normalized to $C = 0.1$)
\item MIGA: 100,000 (normalized to $M = 0.3$)
\item vePARS: 50,000 locked for 2 years (normalized to $V = 0.25 \times \sqrt{0.5} \approx 0.177$)
\item KARMA: 0.6 (active contributor)
\end{itemize}

\textbf{ASHA calculation:}
\begin{align}
\text{Base} &= 0.3 \cdot 0.1 + 0.5 \cdot \sqrt{0.3 \cdot 0.177} + 0.2 \cdot 0.177 \\
&= 0.03 + 0.5 \cdot 0.230 + 0.0354 \\
&= 0.03 + 0.115 + 0.0354 = 0.18 \\
\text{Multiplier} &= 1 + \ln(1 + 0.6 \cdot 2.718) = 1 + \ln(2.631) = 1.97 \\
\text{ASHA} &= 0.18 \times 1.97 = \boxed{0.35}
\end{align}

\textbf{Result:} Despite much less capital, contributor gets 46\% more power than whale due to lock + reputation.

\subsubsection{Example 3: Long-Term Believer}

\textbf{User profile:}
\begin{itemize}
\item CYRUS: 50,000 (normalized to $C = 0.3$)
\item MIGA: 500,000 (normalized to $M = 0.5$)
\item vePARS: 500,000 locked for 4 years (normalized to $V = 0.5 \times 1.0 = 0.5$)
\item KARMA: 0.8 (consistent contributor)
\end{itemize}

\textbf{ASHA calculation:}
\begin{align}
\text{Base} &= 0.3 \cdot 0.3 + 0.5 \cdot \sqrt{0.5 \cdot 0.5} + 0.2 \cdot 0.5 \\
&= 0.09 + 0.5 \cdot 0.5 + 0.1 \\
&= 0.09 + 0.25 + 0.1 = 0.44 \\
\text{Multiplier} &= 1 + \ln(1 + 0.8 \cdot 2.718) = 1 + \ln(3.174) = 2.16 \\
\text{ASHA} &= 0.44 \times 2.16 = \boxed{0.95}
\end{align}

\textbf{Result:} Balanced capital, max lock, and strong reputation yield 4x power of pure whale.

% =============================================================================
\subsection{Parameter Justification}
% =============================================================================

\subsubsection{Weight Selection Rationale}

The weights in the ASHA formula were chosen based on the following principles:

\begin{center}
\begin{tabular}{clp{6cm}}
\toprule
\textbf{Weight} & \textbf{Component} & \textbf{Rationale} \\
\midrule
30\% & CYRUS & Foundation stability matters, but shouldn't dominate \\
50\% & $\sqrt{M \cdot V}$ & Community + commitment is most important factor \\
20\% & vePARS & Additional reward for time-locking \\
\midrule
& \textbf{Total:} 100\% & \\
\bottomrule
\end{tabular}
\end{center}

\textbf{Why geometric mean for MIGA and vePARS?}

\begin{itemize}
\item \textbf{Prevents single-axis optimization:} Cannot max governance by only buying MIGA or only locking PARS
\item \textbf{Enforces balance:} Optimal strategy requires both community participation (MIGA) and commitment (vePARS)
\item \textbf{Diminishing returns:} Doubling one factor increases power by only $\sqrt{2} \approx 1.41\times$
\item \textbf{Mathematical elegance:} Geometric mean is the unique symmetric function with this property
\end{itemize}

\subsubsection{KARMA Multiplier Justification}

\textbf{Why logarithmic instead of linear?}

\begin{itemize}
\item \textbf{Bounded impact:} KARMA can at most double power (2.31x max), preventing reputation dominance
\item \textbf{Diminishing returns:} Encourages broad participation over perfect scores
\item \textbf{Sybil resistance:} Multiplicative (not additive) means capital still required
\item \textbf{Fairness:} Early contributors get proportionally more benefit
\end{itemize}

\textbf{Why use $\ln(1 + K \cdot e)$ specifically?}

\begin{itemize}
\item Passes through $(0, 0)$ and $(1, \ln(1+e) \approx 1.31)$
\item Using $e$ as scaling factor gives nice mathematical properties
\item Shape matches intuitive "helpful but not dominant" reputation curve
\end{itemize}

% =============================================================================
\subsection{Comparison with Other Governance Systems}
% =============================================================================

\begin{longtable}{p{2.5cm}p{3.5cm}p{6.5cm}}
\toprule
\textbf{Protocol} & \textbf{Governance Model} & \textbf{Comparison to MIGA} \\
\midrule
\endhead

\textbf{Compound} &
Token-weighted voting &
\begin{itemize}[leftmargin=*,nosep]
\item Pure plutocracy: 1 token = 1 vote
\item No time-locking, no reputation
\item Vulnerable to whale dominance
\item MIGA adds commitment and reputation factors
\end{itemize} \\[1em]

\textbf{Curve (veCRV)} &
Vote-escrowed tokens &
\begin{itemize}[leftmargin=*,nosep]
\item Time-weighted voting power
\item Linear time scaling (MIGA uses square root)
\item Single token (CRV), no paired lock
\item No reputation system
\item MIGA requires balanced PARS+MIGA and adds KARMA
\end{itemize} \\[1em]

\textbf{Maker (MKR)} &
Token-weighted + delegates &
\begin{itemize}[leftmargin=*,nosep]
\item Delegation allows expertise concentration
\item No time-locking requirement
\item No explicit reputation
\item MIGA uses direct voting with reputation multiplier
\end{itemize} \\[1em]

\textbf{Gitcoin} &
Quadratic voting &
\begin{itemize}[leftmargin=*,nosep]
\item $\text{Power} = \sqrt{\text{tokens}}$ reduces whale power
\item No time-locking
\item Reputation via Passport (similar to KARMA)
\item MIGA combines quadratic elements with time-locking
\end{itemize} \\[1em]

\textbf{Optimism} &
Bicameral (Token + Citizens) &
\begin{itemize}[leftmargin=*,nosep]
\item Separate Token House and Citizens' House
\item Reputation-based citizenship
\item No time-locking for tokens
\item MIGA combines both in single formula (ASHA)
\end{itemize} \\

\bottomrule
\end{longtable}

\textbf{MIGA's unique combination:}

\begin{itemize}
\item \textbf{Multi-token requirement:} CYRUS + MIGA + PARS
\item \textbf{Paired time-locking:} Must lock both MIGA and PARS
\item \textbf{Reputation multiplier:} KARMA amplifies but doesn't dominate
\item \textbf{Diminishing returns:} Multiple geometric/logarithmic factors
\item \textbf{Anti-capture design:} No single factor gives majority power
\end{itemize}

% =============================================================================
\subsection{Governance Attack Vectors and Mitigations}
% =============================================================================

\subsubsection{Whale Attack}

\textbf{Attack:} Single wealthy actor buys majority of tokens to control governance.

\textbf{MIGA mitigations:}

\begin{enumerate}
\item \textbf{Multi-token requirement:} Need CYRUS, MIGA, and PARS (three separate purchases)
\item \textbf{Geometric mean:} Diminishing returns on single-token accumulation
\item \textbf{Bonding curve:} Price increases with purchases (anti-whale mechanism)
\item \textbf{KARMA requirement:} Need reputation multiplier for max power (cannot buy)
\item \textbf{Time-lock requirement:} Capital locked for years, illiquid
\end{enumerate}

\textbf{Result:} Whale attack requires buying three tokens, locking for years, and building reputation. Cost is prohibitive relative to power gained.

\subsubsection{Sybil Attack}

\textbf{Attack:} Create many fake identities to gain reputation without capital.

\textbf{MIGA mitigations:}

\begin{enumerate}
\item \textbf{Multiplicative KARMA:} Reputation alone gives zero power (multiplier of base power)
\item \textbf{Capital requirements:} Need CYRUS, MIGA, PARS to have any power
\item \textbf{Verified contributions:} KARMA requires on-chain proofs and attestations
\item \textbf{Cost of identity:} Creating quality identities is expensive
\end{enumerate}

\textbf{Result:} Sybil attack provides no power without capital. Creating many identities with capital is equivalent to one large identity (no advantage).

\subsubsection{Short-Term Mercenary Attack}

\textbf{Attack:} Borrow tokens, vote for malicious proposal, return tokens.

\textbf{MIGA mitigations:}

\begin{enumerate}
\item \textbf{Time-lock requirement:} Power comes from vePARS, which requires locking tokens
\item \textbf{KARMA requirement:} Reputation cannot be borrowed
\item \textbf{Quadratic timelock voting:} Proposals require sustained consensus over time
\item \textbf{Paired lock:} Must lock both MIGA and PARS (harder to borrow both)
\end{enumerate}

\textbf{Result:} Borrowed tokens provide minimal power. Attack requires borrowing multiple tokens and cannot fake reputation.

\subsubsection{Cartel Formation}

\textbf{Attack:} Multiple actors coordinate to pool governance power.

\textbf{MIGA mitigations:}

\begin{enumerate}
\item \textbf{KARMA is soulbound:} Cannot pool reputation across addresses
\item \textbf{Diminishing returns:} Pooling capital provides sublinear power gains
\item \textbf{Constitutional checks:} Supreme Council and Ten DAOs provide oversight
\item \textbf{Transparency:} All votes are public (privacy-preserving but auditable)
\end{enumerate}

\textbf{Result:} Cartel formation is possible but requires sustained coordination across multiple actors with verifiable reputations. Constitutional mechanisms provide backstop.

% =============================================================================
\subsection{Future Parameter Updates}
% =============================================================================

The ASHA formula parameters may be updated through governance:

\textbf{Updatable parameters:}

\begin{itemize}
\item CYRUS weight (currently 0.3)
\item Geometric mean weight (currently 0.5)
\item vePARS weight (currently 0.2)
\item KARMA scaling factor (currently $e$)
\item Maximum time-lock duration (currently 4 years)
\item KARMA calculation weights
\end{itemize}

\textbf{Update process:}

\begin{enumerate}
\item Guardian+ tier submits parameter change proposal (ASHA $\geq$ 100)
\item Simulation shows impact on existing governance distribution
\item Quadratic timelock voting (7-day voting, 2-day timelock)
\item Requires 66\% supermajority for parameter changes
\item Supreme Council emergency veto (5/9 signatures)
\end{enumerate}

\textbf{Constraints on parameter changes:}

\begin{itemize}
\item Weights must sum to 1.0
\item No single weight can exceed 0.5 (prevents single-token dominance)
\item KARMA multiplier must be logarithmic or sublinear
\item Changes cannot be retroactive
\item Minimum 30-day notice before parameter changes take effect
\end{itemize}

This ensures that the tokenomics can evolve while maintaining anti-capture properties.

% =============================================================================
\subsection{Summary}
% =============================================================================

The MIGA token economics implement a sophisticated governance system that:

\begin{enumerate}
\item \textbf{Requires multi-dimensional participation:} Capital (CYRUS, MIGA), commitment (vePARS), and reputation (KARMA)
\item \textbf{Implements diminishing returns:} Geometric means and logarithmic functions prevent single-axis dominance
\item \textbf{Rewards long-term alignment:} Time-locking amplifies power but with square root scaling for fairness
\item \textbf{Resists capture:} Multiple anti-capture mechanisms make governance attacks prohibitively expensive
\item \textbf{Creates economic flywheel:} Participation $\rightarrow$ Treasury $\rightarrow$ Grants $\rightarrow$ Growth $\rightarrow$ More participation
\item \textbf{Allows evolution:} Parameters can be updated through governance while maintaining anti-capture properties
\end{enumerate}

The ASHA governance power formula represents a novel approach to DAO governance that balances capital, commitment, and contribution while preventing plutocracy and maintaining decentralization.

\begin{tcolorbox}[colback=migadark,coltext=white,title={\textbf{ASHA: Truth Through Alignment}}]
\centering
\textit{``In the MIGA protocol, governance power comes not from wealth alone,\\
but from the combination of capital, commitment, and contribution to the community.''}\\[0.5em]
--- MIGA Protocol Constitution
\end{tcolorbox}
